\chapter{Conclusion and Future Work}

This aim of this project was to create a tool that will aid the formation of strategies to help segments of a population by a local council. The end product was a Django web application which gives different perspectives through different visualizations of a single data set. The tool\textsc{\char13}s usefulness depends on the quality of the data and whether the data represents the population well. According to some members of the policy department in Lambeth Council, it has potential to be used should there be more thought put in what data sets are to be used.\par

The tool may have applications for other local governments dealing who would like to get to know a population more and explore what sorts issues are present within segments of the population. Outside of local government, this could be used for anyone looking to do customer segmentation such as businesses who want to tailor their services to the surrounding residents.

\section{Future work}
\begin{itemize}
	\item The tool is implemented as a Django web application and should there be improvements, the tool could be packaged as such. This could enable the developers in local councils or potential users of the tool to implement it themselves using their own data sets and explore their population accordingly. Some pages could be turned into reusable Django Views class that can input a data set and output a page with the visualizations which could be used in other Django applications.
	\item The user could benefit with more flexibility in terms of which questions are to be included as the only certain questions are shown and which data set is to be used. Including other years\textsc{\char13} data sets could show whether a segment is improving or not.
	\item Utilizing different sizes of geographical divisions such as ONS\textsc{\char13} output areas or super output areas instead of wards could be beneficial in targeting more specific areas.
	\item In other applications of the tool, the data may be more sensitive and require security features.
	\item The comparison of the results on two different questions as graphs or maps could be added as another page. This would enable the user to compare questions rather than groups. It could highlight correlations between clusters as well. For example, a map showing the residents who claim child benefits and another map showing the residents who live in a council flat might show the same trends and could be interpreted as a correlation between the two.
	\item The use of a more performance database such as MongoDB to store the data set could be implemented in future iterations of the program to increase scalability for data sets larger than one those rows.
More statistical information such as mean, median, mode and standard deviation which standard knowledge in local government will provide more insight to the distribution of the data.
\end{itemize}