\chapter{Professional and Ethical Issues}

The piece of software produced is a combination of my work and the work of Third Parties, which include Open Source libraries such as Django (citation), Graphos (citation), Google Maps (citation), Google Charts (citation), Kmodes (citation), and others (citation). These libraries were utilized to take advantage of existing work however, they were adapted to the purposes of this project. Relevant libraries and sources are cited in both source code and the description in the implementation to give credit to third party content.
The main data used, the 2016 Lambeth Residential Survey, keeps its participants anonymous and according Lambeth has no security issues in terms of publicizing the data. The software places a resident under an anonymous serial, no name of any sort is saved nor a specific address. The software maintains the anonymity of the location of their residence at the Ward level instead of a more specific post code level.\par

The clustering of the survey data used a third-party software. Though sensitive information such as race and sex is included as variables in which to cluster, as far as I know the clustering algorithm used do not discriminate on those variables. \par

During the evaluation stage, a demonstration of the software was conducted to a group of 4 people from the policy department Lambeth Council to gain their feedback. The demonstration and the acceptance of their feedback was done respectfully as they have given honest criticisms of my work. \par
