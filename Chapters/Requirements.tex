\chapter{Requirements} \label{ch:requirements}

The functional and non-functional requirements was created based on communication, through interviews and email, with a member of the Policy and Communications Team, and requirements which I have created based on initial requirements by our contact. Our contact described the context in which the tool may be used, the overall objective of the software, the users\textsc{\char13} technical abilities and some functional requirements, namely the method to create groups (see section \ref{creating_groups}). Communication with our contact was not sustained after initial requirements were collected due to the council\textquotesingle s large workload. The rest of the requirements was created in accordance with the overall goal, initial requirements and according to my own analysis of the data.

\section{Objective}
The purpose of this application is to support those in local government councils, in charge of creating policies, to explore data on residents demographics and satisfaction with council services. User-defined groups will allow the user to focus on specific portions of the population. Aided by visualizations through graphs and maps, it may lead the user to identify which segments of the population requires the which resources.

\section{Description of data to be used}
The following local data provided by Lambeth should be used:
\begin{itemize}
  \item \textbf{CSV of Lambeth\textquotesingle s 2016 Residential Survey}: a survey of a sample of Lambeth\textquotesingle s population consisting of 1024 people about their quality of life, what they thought of Lambeth\textquotesingle s services and about the respondents themselves. The data is in the form of a CSV text file. Residents\textquotesingle  answers are categorical data in the form of code specified by the Survey Code Translation (see point below, Survey Code Translation). Single answer questions have its own column and its choice code is a positive integer. Multiple answer questions, each answer is regarded as a sub question in the form of `Q5A' meaning question 5, choice code `A' (see below Survey Code Translation). Therefore, a sub question has its own column and entries are in code as a 1 or 0 meaning yes or no respectively. There are also other fields which have been added to the original survey results such as group, subgroup, quintile, which is a result of previous data analysis.
  \item \textbf{Microsoft Word document of Lambeth\textquotesingle s 2016 Residential Survey Code Translation}: a Microsoft Word document of the original survey with the original questions and code used in the data associated with the question. Under each question is the choice code and English meaning of the choice. The choice code is either a number for single answer questions (e.g. 1. Male, 2. Female) or letters for multiple answer questions (e.g. A. Access to nature, B. Activities for teenagers, etc.).
  \item \textbf{GeoJSON of Lambeth\textquotesingle s Ward\textquotesingle s Boundary Specifications}: downloaded from Lambeth\textquotesingle s website, it is a GeoJSON, simple geographical features encoded as a JSON, of the geographical boundaries of each ward.
\end{itemize}

\section{Functional Requirements}
Functional requirements with the words textbf{should} or textbf{must} is a required feature. Requirements with the word textbf{may}, is a desirable feature which may or may not be implemented should there not be enough development time.


\begin{enumerate}
  \item Data storage requirements
    \begin{enumerate}
      \item 1.	The system must save the actual files of the datasets (CSV, text files or JSON) and not access the file through a URL of another website.
    \end{enumerate}
  \item Grouping requirements
    \begin{enumerate}
		\item The user should be able to create groups based on the parameters to the 5 factors (see section \ref{creating_groups}). A group\textquotesingle s information should be kept in a database.
		\item The user should be able to view data about a group through graphs.
			\begin{enumerate}
				\item Single answer questions or questions which requires only one answer should be visualized as stacked bar charts or pie charts.
				\item Multi-code questions or questions which requires more than one answer should be visualized as bar charts.
				\item Clicking data on a graph should display which wards have answered the selected question and choice on the map (see functional requirement 4).
				\item The questions under section Questions to be visualized should be visualized following requirements 3a and 3b.
			\end{enumerate}
		\item Like a population density map, which shows higher density concentrations of people as a darker color and lower density concentrations as a lighter color on particular areas of the map, the map should show the selected data (see functional requirement 3c). The map should be divided into the council\textquotesingle s wards; the boundaries of each ward should be obvious. The color of the ward should depend on the number of residents who have the selected question and answer (see functional requirement 3c).
		\item The user should be able to compare values of a data variable between all groups such that groups should be compared to the average value of the variable in the whole population (i.e. compare the percentage of disabled people who are male to the percentage of the whole population who are male).
		\item 1.	The main data used in the visualizations should be the CSV of Lambeth\textquotesingle s Residential Survey and its Code Translation.
	\end{enumerate}
	\item Clustering requirements
	\begin{enumerate}
		\item The system should segment a group using a clustering algorithm.
		\item The system should display any statistical information (e.g. the mean of the answers of a question) on the differences between the clusters.
		\item The user should be able to compare information between the group\textquotesingle s data and each of the clusters\textquotesingle  data.
	\end{enumerate}
\end{enumerate}


\section{Creating groups} \label{creating_groups}
A group should be based on residents\textquotesingle  answers to 5 questions in the residential survey. At least 1 question is required to form a group, therefore the rest of the 4 questions could have any answer. The group will be composed of residents who match the required answer(s) to the required questions.\par

There are 5 questions in which a user can create a group:

\begin{enumerate}
	\item Whether they have a disability or long term illness (Q43)

	\item What type of benefits do they acquire (Q45A-Q45K)
	\item Educational/employment activity (Q46)
	\item Whether they are on the London Living Wage (Q47)
	\item Housing tenure (i.e. council tenant, private owner, etc.) (Q35)
\end{enumerate}

*Question number in the CSV residential survey is in brackets.


\section{Questions to visualize}
Once a group has been created, the user must be able to see the answers to the following survey questions:
\begin{enumerate}
	\item What matters most to them most (Q5) (Multiple, top 3)
	\item How was their last contact with the Council was made (Q26A- Q26G) (Multiple)
	\item How they use the website (Q29)  (Multiple)
	\item What services they have used (Q39) (Multiple)
	\item How they access the internet (Q50) (Multiple)
	\item How well the changes have benefited them (Q11) (Single)
	\item Whether they feel they belong to their neighborhood (Q13\textunderscore R1) (Single)
	\item Whether they value the friendships in their neighborhood (Q13\textunderscore R2) (Single)
	\item Whether they could approach a neighbor for advice (Q13\textunderscore R3) (Single)
	\item Whether neighbors help out each other (Q13\textunderscore R4) (Single)
	\item Whether they would be willing to work with others to improve their neighborhood (Q13\textunderscore R5) (Single)
	\item Whether they would join community events in their area (Q13\textunderscore R6) (Single)
	\item Whether they regularly stop and talk to people in their neighborhood (Q13\textunderscore R7) (Single)
	\item Whether they would speak highly of their neighborhood when asked (Q13\textunderscore R8) (Single)
	\item Gender (QGEN) (Single)
	\item Age (QAGE) (Single)
	\item Ethnicity (QETH) (Single)
\end{enumerate}

*Question number in the residential survey is in brackets followed by whether they are a single answer question (Single) or multiple answer question (Multiple).\par

The answers to the questions above should be visualized using graphs for each group and their clusters based on functional requirement 3. 

\section{Non-functional requirements}
\textbf{Usability}: The users for the program are the policy makers of the council, those responsible for formulating strategies to allocate resources for a council based on the presentation of data analysis given to them. They do not necessarily have technical skills, in terms of being able to operate applications, or advanced statistical analysis skills. Since the users are not technical savvy, the ease of use is imperative to the design of the user interface. The visualizations should also be easily interpreted. \par

In terms of maintainability, this version of the system is only a prototype to show the potentials of such a system, therefore there will not be a need for maintainability of the code. In terms of security, the data being used is anonymous therefore there will be no need for security measures for the data.


