%%%%%%%%%%%%%%%%%%%%%%%%%%%%%%%%%
% 6CCS3PRJ Final Year Individual Project Report
% jan_aldous.torres@kcl.ac.uk
%%%%%%%%%%%%%%%%%%%%%%%%%%%%%%%%%
\documentclass[11pt]{informatics-report}
\usepackage{color}
\usepackage[square,sort,comma,numbers]{natbib} %References
\usepackage{graphicx}
\graphicspath{ {images/} }
\usepackage{textcomp}
\usepackage[strings]{underscore}

%%%%%%%%%%%%%%%%%%%%%%%%%%%%%%%%%
% Front Matter - project title, name, supervisor name and date
%%%%%%%%%%%%%%%%%%%%%%%%%%%%%%%%%
\title{A tool for the visualization of populations}
\author{Jan Aldous Torres}
\studentID{1323454}
\supervisor{Dr Simon Miles}

\date{\today}

\abstractFile{FrontMatter/abstract.tex}
\ackFile{FrontMatter/acknowledgements.tex} %Remove line if you do not want acknowledgements

\begin{document}
\createFrontMatter
\onehalfspacing
\tableofcontents
\listoffigures
\doublespacing

%%%%%%%%%%%%%%%%%%%%%%%%%%%%%%%%%
% Report Content
%%%%%%%%%%%%%%%%%%%%%%%%%%%%%%%%%
% You can write each chapter directly here or in a separate .tex file and use the include command.

\chapter{Introduction}

Local government councils are challenged with providing services using limited resources. They also need to cater to a diverse population with different needs. Therefore, there is a need to create policies which efficiently utilize and allocate resources in the most effective way. \par

Using data science in policy decision making is gaining in popularity. Data science provides insights on trends and patterns about customer needs and behavior which may lead to policies to better cater to those needs. One technique to identify customer needs is customer segmentation which groups individuals with similar needs. This will let councils better understand the needs of segments of the population thus letting them tailor services to provide better services and identify groups which require the most attention. Clustering algorithms, which also group similar objects together, can automate the identification of segments which is especially useful when one deals with large data sets. Since it uses a similarity calculation, it could identify groups which have not been considered before. The national government and the Local Government Association have promoted customer segmentaion in local governments especially after budget cuts \cite{smartcities} \cite{lgaguide}. \par

The process of data analysis often involves technical skills since data analysis tools are sophisticated tools. Policy makers may not have these skills. This distances themselves from interacting and understanding the data.\par

This project aims to produce a software that allows policy makers in councils to explore the data themselves. This project aims to implement a prototype of such a tool as a Django web application.\par

The tool allows the user to create population groups and visualize data about them. Visualizations of this data will also help to compare characteristics of the group to the wider population and subgroups. This comparison may lead to observations on the circumstances a group is facing. This may lead to the conclusion that the council may want to increase or tailor resources allocated to a group or service. Since the user may not necessarily know what they are looking for when they view the data, the visualizations will be in the form that will let the user explore more aspects of the data. \par

The remainder of this paper is organized as follows. Chapter \ref{ch:background} introduces the concept of customer segmentation, clustering, data exploration and related work. Chapter \ref{ch:requirements} discusses the requirements of the tool. Chapter \ref{ch:design} outlines the specification and design of the tool. Chapter \ref{ch:implementation} describes how the tool was implemented. Chapter \ref{ch:professionalissues} discuses the professional and ethical issues. Chapter \ref{ch:evaluation} describes the results of the evaluation after the tool was demonstrated to members of the policy department. Lastly, chapter \ref{ch:conclusion} discuses the conclusions and future work of the project.

\chapter{Background} \label{ch:background}
This chapter includes a background on customer segmentation and clustering algorithms, followed by a background on the use of data analysis, data exploration. The chapter ends with a section on related applications to the software to be created in this project.\par

\section{Customer segmentation}

Customer segmentation or market segmentation tool is widely used in the marketing field. The whole market is divided such that the each segment has similar ``needs, wants or demand''. The organization uses it to target a segment to tailor services to their needs.\par

The four basic market segmentation-strategies: behavioral, based on a person\textsc{\char13}s behavior and decision-making process, demographic, based on the background of an individual such as age, gender, nationality, psychographic, similar to behavioral but is more about the lifestyle and interests, and geographical, based on a person\textsc{\char13}s location \cite{marketing912017}. Local governments usually uses demographic segmentation more than the other types \cite{davey2009}. There are other more complex segmentation approaches involving the customer\textsc{\char13}s interaction with a service or access to a service. However, they are not as prevalent in the public sector \cite{davey2009}. Clustering algorithms could also be used to identify the segments \cite{lgaguide}. \par

The resources of an organization are limited and it usually cannot cater for the whole population. The private and public sector have to approach customer segmentation differently as the private sector aims to maximize business and targets the specific segments of the population. This is contrary to the public sector which has the mandate to cater to the whole population. They focus more on offering appropriate services to segments of the population. Policies usually targets segments of the population, where the minorities may not experience the benefits and only the majority of the population do.\par

Customer segmentation is a concept promoted by the national government and the Local Government Association (LGA) has created a guidance document for any council that wishes to implement such a tool \cite{lgaguide}. They give guidance on which data sets to use, segmentation of the population with k-means clustering and ways to visualize the data. Explicit data about its customers is combined with implicit data ``knowledge of staff on customers using a service'' \cite{smartcities}. This would give a more complete view on not only the customers\textsc{\char13} profile but also their behaviors in interacting with the service.\par

\section{Clustering}

Clustering is a task which aims to group data points in an unlabeled data set where each point within a group, called a cluster, is more similar to another point in another cluster. It can be implemented as different algorithms using varying methods of determining similarity. The number of clusters is determined by the user depending on their expectations from the data. It has numerous applications in different fields such as data science, biology, medicine and business to name a few. It is also a technique used in customer segmentation where a clustering algorithm can label the segments within a collection of market data \cite{lgaguide}. \par


The optimal number of clusters is subjective however, there are ways to determine the best number of clusters to use. One method, the elbow method, requires a graph of the percentage of variance and the number of clusters \cite{thedatasciencelab2014}. The sum of the intra-cluster distances between points in a cluster. The normalized intra-cluster sum of squares gives the variance quantity. The percentage variance is calculated as the ratio of the between-group variance to the total variance. As the number of clusters increases the percentage of variance increases significantly until a certain point, which creates a bend. In this method, the clearest bend in the graph determines most suitable number of clusters, however finding the bend may be an ambiguous task as it depends on the clarity of the bend.\par

There are different measures to determine if data points should form a cluster which include the structure of a cluster through closeness but also the concept of the cluster. Distance, metrics such as Euclidian, cosine, Jaccard, Hamming distance, Manhattan \cite{oneilschutt2014}. An implementation of this is the k-means algorithm which uses Euclidian distance to calculate similarity. However, it is not as preferable for clustering categorical data. Euclidian distance calculates a mean which could relate two discrete values or categories together. Huang\textsc{\char13}s k-modes \cite{Huang1998} is an algorithm to avoid this issue. It instead matches categorical data of different data points through the calculation of the mode, respecting the categories as discrete values, rather than the contrary in Euclidean distance. \par

There are other ways of clustering through structure which involve density includes a point such that it does not exceed density of points within the radius of a cluster \cite{tutorialspoint}. Conceptual clustering uses a descriptive language to define the concepts of the cluster through instead of similarity measures \cite{clusteringintroduction}. The descriptive language defines the model in which to compare the similarity of data points.\par

\section{Data science}
Data science is a multi-disciplinary field which intends to ``analyze and understand actual phenomena'' \cite{Hayashi1998} but it also deals with the communication of its results \cite{oneilschutt2014}. It is multi-disciplinary because it not only involves statistical data analysis, but also domain expertise, data visualization, machine learning which are skills needed in this field.\par

In the field of data science, data exploration is one of the steps in the data science process. After data is collected from the real world, processed and cleaned, which converts raw data into a usable format. Depending on the process used, the next step is either data exploration \cite{oneilschutt2014} or stating the question followed by data exploration \cite{peng2016}.\par

Nevertheless, though the order may be different, both process agree that data science involves exploring the data, building models, interpreting and then communicating the results. These steps demonstrate the need for different disciplines in this field as statistics and domain expertise is required in building models and data visualization is required for communicating the results.\par

\subsection{Data exploration}
Data exploration aids the user in understanding the structure of the data set, discovering if there are missing values, examining the distributions of individual variables, to name a few. O'Neil and Schutt \cite{oneilschutt2014} explains that data exploration is not only about confirming expectations or hypotheses but also to discover new information which one did not expect. Tools include plots, graphs and summary statistics. Data visualization through plots such as boxplots and graphs allows one to absorb information and identify patterns more easily. Summary statistics involve simple statistical data such as a variable\textsc{\char13}s mean, median, mode and range.

\section{Related works}

This subsection describes some works related to the software created in this project, which includes academic research of prototypes for data exploration, customer segmentation tools and government websites which visualize data. There are other related tools, however, reviewing all of them is beyond the scope of this report.
\par

\subsection{Data exploration tools}

Gogolou et al. \cite{gogoloudata2016} created prototype, Data Curation and Validation (DCV) which supports user-data interaction especially in data cleaning and exploration. DCV, originally a data cleaning tool, has lead them to believe that any interaction with the data is data exploration. Therefore, the tool has been expanded to include data exploration features. There are three components, the presentation, visualization and profiling. The presentation component allows the user to select the type of visualization they wish to see on a variable. The profiling component predicts if there is erroneous data such as outliers and recommends actions to the user according to previous data analysis. The activity component deals with data exploration tasks such as data search and analysis.\par

Similarly, Graves and Hendler \cite{gravesvisualization2013} created a prototype which allows users with no technical expertise, but more specifically targeted to visualize open government data (OGD). Ordinary citizens, journalists, those with no experience in consuming OGD are at a disadvantage. Visualizations enables an ordinary person to consume large amounts of data. Therefore, creating visualizations of OGD will enable them to consume this data. They identified groups of software which deal with this problem namely, office suits, business intelligence software, specialized analysis tools and visualization APIs which are not suitable for the consumption of OGD by the layman. They discuss what is needed in such a software: simplification of the visualization process, providing meta-data about the current data set such as location, time of creation, understandable conventions on description of data (i.e. turn FY1998 into its worded description), including the contact information of the provider, sharing visualizations on social networks.\par

The neighborhood statistics website \cite{neighbourhoodstatistics} of the UK government gives a summary and visualizes some census data. The website enables the user to view census data on a geographical area. The user can view a data set as a table or on a map (see figure \ref{fig:neighborhood}). The table shows aggregation of the data set where each row represents a variable. The map visualizes the values of the selected variable through different shades of color according to the legend. A time series graph also allows the user to view differences of the aggregate value over the years. The tool however limits the user to viewing only one variable at a time.\par

\begin{figure}[h]
\centering
\includegraphics[scale=0.3]{neighborhood}
\caption{Screenshot of a page in neighbourhood.statistics.co.uk}
\label{fig:neighborhood}
\end{figure}

\subsection{Customer segmentation tools}

There are tools which presents pre-made customer segmentation analysis of the UK population like Mosaic Public Sector by Experian \cite{experian}, Acorn by CACI \cite{acornuserguide} and Kent \& Medway\textsc{\char13}s interactive guide \cite{kentmidway}. \par

Mosaic Public Sector uses government data and visualizes this data through an interactive tool intended for public sector use. The tool has pre-made customer groups using their data and the visualizations include maps, photos and graphs. It utilizes data from different sources, from census to social media data to create the segments. \par

On the other hand, Acorn uses government but also commercially available data. Like Mosaic, it has created pre-made customer segments. However, there is a feature that enables the user to find out more about their customer through just their post code. This approach separates itself from matching demographic classifications to locations to matching locations to demographic types. \par

Kent \& Medway tool which visualizes the segments its population by social class determined by the council\textsc{\char13}s own customer segmentation analysis. It has a feature which compares a data variable between all groups of the population (see figure \ref{fig:kent}).\par

There have been similarities between implemented tools and recommendations to differentiate and describe each group. Kent \& Medway suggested to include maps showing the concentration of that group in each ward and in addition to a textual description, they include pictures to describe each group. It also includes graphs to visualize data but also word maps of key characteristics. The LGA guide \cite{lgaguide} has suggested that the use of spider diagrams which detail the variables compared to the town and district average. Similarly Kent \& Medway and Acorn, they graph pieces of data with an index of the group compared to the overall population. This visualizes both the average value and the groups relation to that average displays whether they are higher or lower than that average however each graph is created for each variable.


\begin{figure}[h]
\centering
\includegraphics[scale=0.3]{kent}
\caption{Screenshot of a page in Kent \& Medway interactive site}
\label{fig:kent}
\end{figure}
\chapter{Requirements}

The functional and non-functional requirements was created based on communication, through interviews and email, with a member of the Policy and Communications Team, and requirements which I have created based on initial requirements by our contact. Our contact described the context in which the tool may be used, the overall objective of the software, the users\textsc{\char13} technical abilities and some functional requirements, namely the method to create groups (see section \ref{creating_groups}). Communication with our contact was not sustained after initial requirements were collected due to the council\textquotesingle s large workload. The rest of the requirements was created in accordance with the overall goal, initial requirements and according to my own analysis of the data.

\section{Objective}
The purpose of this application is to support those in local government councils, in charge of creating policies, to explore data on residents demographics and satisfaction with council services. User-defined groups will allow the user to focus on specific portions of the population. Aided by visualizations through graphs and maps, it may lead the user to identify which segments of the population requires the which resources.

\section{Description of data to be used}
The following local data provided by Lambeth should be used:
\begin{itemize}
  \item \textbf{CSV of Lambeth\textquotesingle s 2016 Residential Survey}: a survey of a sample of Lambeth\textquotesingle s population consisting of 1024 people about their quality of life, what they thought of Lambeth\textquotesingle s services and about the respondents themselves. The data is in the form of a CSV text file. Residents\textquotesingle  answers are categorical data in the form of code specified by the Survey Code Translation (see point below, Survey Code Translation). Single answer questions have its own column and its choice code is a positive integer. Multiple answer questions, each answer is regarded as a sub question in the form of `Q5A' meaning question 5, choice code `A' (see below Survey Code Translation). Therefore, a sub question has its own column and entries are in code as a 1 or 0 meaning yes or no respectively. There are also other fields which have been added to the original survey results such as group, subgroup, quintile, which is a result of previous data analysis.
  \item \textbf{Microsoft Word document of Lambeth\textquotesingle s 2016 Residential Survey Code Translation}: a Microsoft Word document of the original survey with the original questions and code used in the data associated with the question. Under each question is the choice code and English meaning of the choice. The choice code is either a number for single answer questions (e.g. 1. Male, 2. Female) or letters for multiple answer questions (e.g. A. Access to nature, B. Activities for teenagers, etc.).
  \item \textbf{GeoJSON of Lambeth\textquotesingle s Ward\textquotesingle s Boundary Specifications}: downloaded from Lambeth\textquotesingle s website, it is a GeoJSON, simple geographical features encoded as a JSON, of the geographical boundaries of each ward.
\end{itemize}

\section{Functional Requirements}
Functional requirements with the words textbf{should} or textbf{must} is a required feature. Requirements with the word textbf{may}, is a desirable feature which may or may not be implemented should there not be enough development time.


\begin{enumerate}
  \item Data storage
    \begin{enumerate}
      \item 1.	The system must save the actual files of the datasets (CSV, text files or JSON) and not access the file through a URL of another website.
    \end{enumerate}
  \item Grouping requirements
    \begin{enumerate}
		\item The user should be able to create groups based on the parameters to the 5 factors (see section \ref{creating_groups}). A group\textquotesingle s information should be kept in a database.
		\item The user should be able to view data about a group through graphs.
			\begin{enumerate}
				\item Single answer questions or questions which requires only one answer should be visualized as stacked bar charts or pie charts.
				\item Multi-code questions or questions which requires more than one answer should be visualized as bar charts.
				\item Clicking data on a graph should display which wards have answered the selected question and choice on the map (see functional requirement 4).
				\item The questions under section Questions to be visualized should be visualized following requirements 3a and 3b.
			\end{enumerate}
		\item Like a population density map, which shows higher density concentrations of people as a darker color and lower density concentrations as a lighter color on particular areas of the map, the map should show the selected data (see functional requirement 3c). The map should be divided into the council\textquotesingle s wards; the boundaries of each ward should be obvious. The color of the ward should depend on the number of residents who have the selected question and answer (see functional requirement 3c).
		\item The user should be able to compare values of a data variable between all groups such that groups should be compared to the average value of the variable in the whole population (i.e. compare the percentage of disabled people who are male to the percentage of the whole population who are male).
		\item 1.	The main data used in the visualizations should be the CSV of Lambeth\textquotesingle s Residential Survey and its Code Translation.
	\end{enumerate}
	\item Clustering
	\begin{enumerate}
		\item The system should segment a group using a clustering algorithm.
		\item The system should display any statistical information (e.g. the mean of the answers of a question) on the differences between the clusters.
		\item The user should be able to compare information between the group\textquotesingle s data and each of the clusters\textquotesingle  data.
	\end{enumerate}
\end{enumerate}


\section{Creating groups} \label{creating_groups}
A group should be based on residents\textquotesingle  answers to 5 questions in the residential survey. At least 1 question is required to form a group, therefore the rest of the 4 questions could have any answer. The group will be composed of residents who match the required answer(s) to the required questions.\par

There are 5 questions in which a user can create a group:

\begin{enumerate}
	\item Whether they have a disability or long term illness (Q43)

	\item What type of benefits do they acquire (Q45A-Q45K)
	\item Educational/employment activity (Q46)
	\item Whether they are on the London Living Wage (Q47)
	\item Housing tenure (i.e. council tenant, private owner, etc.) (Q35)
\end{enumerate}

*Question number in the CSV residential survey is in brackets.


\section{Questions to visualize}
Once a group has been created, the user must be able to see the answers to the following survey questions:
\begin{enumerate}
	\item What matters most to them most (Q5) (Multiple, top 3)
	\item How was their last contact with the Council was made (Q26A- Q26G) (Multiple)
	\item How they use the website (Q29)  (Multiple)
	\item What services they have used (Q39) (Multiple)
	\item How they access the internet (Q50) (Multiple)
	\item How well the changes have benefited them (Q11) (Single)
	\item Whether they feel they belong to their neighborhood (Q13\textunderscore R1) (Single)
	\item Whether they value the friendships in their neighborhood (Q13\textunderscore R2) (Single)
	\item Whether they could approach a neighbor for advice (Q13\textunderscore R3) (Single)
	\item Whether neighbors help out each other (Q13\textunderscore R4) (Single)
	\item Whether they would be willing to work with others to improve their neighborhood (Q13\textunderscore R5) (Single)
	\item Whether they would join community events in their area (Q13\textunderscore R6) (Single)
	\item Whether they regularly stop and talk to people in their neighborhood (Q13\textunderscore R7) (Single)
	\item Whether they would speak highly of their neighborhood when asked (Q13\textunderscore R8) (Single)
	\item Gender (QGEN) (Single)
	\item Age (QAGE) (Single)
	\item Ethnicity (QETH) (Single)
\end{enumerate}

*Question number in the residential survey is in brackets followed by whether they are a single answer question (Single) or multiple answer question (Multiple).\par

The answers to the questions above should be visualized using graphs for each group and their clusters based on functional requirement 3. 

\section{Non-functional requirements}
\textbf{Usability}: The users for the program are the policy makers of the council, those responsible for formulating strategies to allocate resources for a council based on the presentation of data analysis given to them. They do not necessarily have technical skills, in terms of being able to operate applications, or advanced statistical analysis skills. Since the users are not technical savvy, the ease of use is imperative to the design of the user interface. The visualizations should also be easily interpreted. \par

In terms of maintainability, this version of the system is only a prototype to show the potentials of such a system, therefore there will not be a need for maintainability of the code. In terms of security, the data being used is anonymous therefore there will be no need for security measures for the data.



\chapter{Specification and Design} \label{ch:design}

This section describes the design of the system as a website.

\section{Platform}
A web application as a platform to create visualizations takes advantage of the numerous open source visualization libraries and HTML, CSS and JavaScript are well equipped in creating interactive and visually pleasing interfaces. For these reasons, the application will take the form of a website.\par

More specifically the design will be implemented as a Django \cite{django} web application. Django is a popular Python web application. It is well documented and has numerous open source libraries which supports the creation of visualizations. Using this framework, third-party libraries could be utilized to produce JavaScript needed to create visualizations. Python\textquotesingle s machine learning libraries will be used to manipulate and analyze data.

\section{Preprocessing of data} \label{preprocessing}
Prior to the creation of the application, some preprocessing of the data must be done. The Microsoft Word document of the survey code translation in its current form is not able to be queried since it is just a Word document. It should be created as a formatted text file that is still readable for the user so that the system can query it and and in the future, the user can create the text file themselves. Inputting the translations in one file will be more user-friendly than inputting translations of individual questions through a normal form.\par

For this system design, since the user has only given the survey code translation as a Word document. I will create the formatted text file myself.\par


\begin{figure}[h]
\centering
\includegraphics[scale=0.5]{textformatted}
\caption{Formatted text file of survey code translation (See Appendix for actual text)}
\end{figure}

As  seen above the text file is still readable as each question\textquotesingle s data is separated by a `\$' sign and each code is separated by a `;' or `|'. The question text is distinguishable from the choices text.

\section{System architecture}

\begin{figure}[h]
\centering
\includegraphics[scale=0.5]{FYP_Architecture}
\caption{System architecture diagram}
\end{figure}

The system will follow the three-layer architecture. \par
The data layer will consist of the following components:
\begin{itemize}
  \item \textbf{Ward}: a GeoJSON file containing the geographical boundaries of each ward
  \item \textbf{Survey Code Translation}: a text file formatted according to section \ref{preprocessing} containing the translations of the survey code to English text
  \item \textbf{Residential data}: a CSV file of the 2016 Lambeth Residential Survey
  \item \textbf{DB}: the database which includes Groups created by the user and groups\textsc{\char13} Clusters produced by the clustering algorithms (see  Appendix for more details)
\end{itemize}

The logic layer will consist of the following components:
\begin{itemize}
  \item \textbf{Clustering library}: a third-party library used by the Clustering component to cluster group data
  \item \textbf{Clustering}: performs clustering on a group and stores results in the database
  \item \textbf{Grouping}: creates groups and extracts groups from survey data
\end{itemize}

The presentation layer will consist of the following components (see section \ref{interface} for more details):
\begin{itemize}
  \item \textbf{Resource views}: shows visualizations data on each group for each question in the residential data set
  \item \textbf{Visualization library}: a third-party library used to create graph and map visualizations
  \item \textbf{Group views}: shows visualizations of groups and compares groups
  \item \textbf{Home}: main menu to choose which group to view
\end{itemize}

Though the original source of the Wards GeoJSON is from Lambeth\textquotesingle s open data website, it will be downloaded as a JSON file and stored as part of the source code of the application. Since the ward boundaries is unlikely to changing, storing the file in the source code is preferable as it will not depend on another server\textquotesingle s performance.\par

The Clustering component could be implemented separately and another clustering algorithm can be implemented without changing the Grouping component. 



\section{Algorithms}
Data analysis will be carried out by the system through data clustering. The clusters will be generated by the Clustering component and will be saved to the database. The default number of clusters is 1 and the user will be able to manipulate the number of clusters. \par

There are numerous clustering algorithms however, due to the categorical nature of the data, k-modes clustering will be used instead of the the usual k-means clustering. K-modes has been proven to be more effective in clustering categorical data (citation).

\section{User interface} \label{interface}
The system\textquotesingle s main goal is to present data and let the user interact with the user in an effective way. The following figures describe the interaction and design of the user interfaces.

\begin{figure}[h]
\centering
\includegraphics[scale=0.5]{FYP_InteractionDiagram.png}
\caption{Interaction diagram of the web application}
\end{figure}

The interaction diagram above describes the flow between the pages in the website. It is designed so that switching pages is done in the least number of clicks possible. This will be done through a navigation bar where most pages will be accessible. The group detail pages will only be accessible from the group\_detail.html page to avoid confusion with other groups\textsc{\char13} pages.\par

The following describes the functions of each page:
\begin{itemize}
  \item \textbf{index.html}: contains a list of links to all groups
  \item \textbf{group\_detail.html}: has visualizations of a group\textquotesingle s data through maps and charts (see section for more detail)
  \item \textbf{group\_compare.html}: compares all the groups to the average of the whole population of a data variable. This will visualize the comparison of the values of the cluster to the average value for that variable in the whole population (see section for more detail)
  \item \textbf{cluster\_compare.html}: compares the data on the clusters of a group, to the group and the whole population for each question (see section for more detail)
  \item \textbf{cluster\_stats.html}: lets the user change the number of clusters and shows differences, namely the means of an answer to a question, between each cluster (see section for more detail)
  \item \textbf{group\_new.html}: contains a form that adds a new group to the database
  \item \textbf{resources\_list.html}: lists links to residential survey questions
  \item \textbf{resources\_datafields.html}: shows visualizations of data on all groups for a survey question
\end{itemize}

*Note that variable is to denote the selected question and choice (i.e. Q11, Full-time employee) \par

This design gives the user different perspectives on a population. Namely, a group\textquotesingle s data (group\_detail.html), data of clusters within a group (cluster\_detail.html), differences between a group and its clusters (cluster\_compare.html, cluster\_stats.html) and differences between groups (resources\_datafields.html). These features are intended to enable the user to identify interesting trends within a specific group, groups or cluster which may not be possible with fewer interfaces. 

\section{Interface design} \label{sec:interfacedesign}
The pages in the interaction diagram are described in more detail in the following page designs. Since a large part of the system is about the presentation of data, the layout of the visualizations and interactions within pages, shown as annotations, were designed.

\begin{figure}[h]
\centering
\includegraphics[scale=0.3]{FYP_Screenshots1.png}
\caption{Page design for group\_detail.html and cluster\_compare.html}
\label{fig:FYP_Screenshots1}
\end{figure}

The page on figure \ref{fig:FYP_Screenshots1} is intended for the user to view the residential data which is visualized in graphs on the right-hand side of the page. Selecting a variable (i.e. a survey question and choice) on a graph will show its data on the map, on the left-hand side of the page, as a population density map. Each ward will be colored depending on the number of residents for which the resident chose the variable. This shows the user where residents who have chosen the variable live. Supporting the map is a bar chart of the number of residents for each ward who have chosen the variable which is ordered from greatest to least. This lets the user easily identify the distribution of the selected variable throughout the wards. The charts will follow functional requirements 3, which specifies which type of graph (e.g. bar chart, pie chart, etc.) to use for which type of question.\par

In addition, the meta-data about the data set used in the page is shown in the top left panel, which shows the number of residents in the group as a raw number and percentage and the source of the data.

\begin{figure}[h]
\centering
\includegraphics[scale=0.5]{comparison}
\caption{Chart for comparison of the data of the whole population, a group and its clusters}
\label{fig:comparison}
\end{figure}

The same page design is used for cluster\_compare.html except that the Question graphs will be in the form of stacked bar charts which show the question data on the whole population, the group and the clusters (see figure \ref{fig:comparison}). \par

The figure \ref{fig:comparison} is the visualization for the resources\_datafields.html page. However, the y-axis will instead be the groups in the database.

\begin{figure}[h]
\centering
\includegraphics[scale=0.3]{FYP_Screenshots2}
\caption{Page design for group\_compare.html}
\label{fig:FYP_Screenshots2}
\end{figure}

The page on figure \ref{fig:FYP_Screenshots2} will let the user compare the percent difference from the whole population using the following equation:\par

(Percent of the group\textquotesingle s population who answered selected question with selected choice-Percent of the overall population who answered selected question with selected choice)*100\par

Therefore groups more than 0\% has a bigger proportion of their population which have answered the selected variable and vice versa for groups less than 0\%. This may be an interesting visualization for some questions more than others.

\begin{figure}[h]
\centering
\includegraphics[scale=0.3]{FYP_Screenshots3}
\caption{Page design for cluster\_stats.html}
\label{fig:FYP_Screenshots3}
\end{figure}

The page on figure \ref{fig:FYP_Screenshots3} gives another perspective to the comparisons show in cluster\_compare.html. Instead of showing the all the data variables, the set of answers are consolidated into a mean. The user can compare means of a question for each cluster which will highlight the differences between each cluster which is not always obvious in cluster\_compare.html. The user may go back to cluster\_compare.html to the data in more detail.

\chapter{Implementation and Testing} \label{ch:implementation}
This chapter explains the development approach used followed by how each element of the design was implemented.


\section{Development approach} 
\label{sec:developmentapproach}
The software was developed through evolutionary prototyping. Since the initial requirements were not complete, this approach gave flexibility on what is to be implemented. Unlike throwaway prototyping where code is thrown away, this method begins with the requirements that are understood and seeks to incrementally implement additional features proposed by the user (citation). Though the usual definition is to modify the software based on feedback from the user, I have taken the place of the user and have incrementally designed and then implemented what is most likely suitable for the users\textsc{\char13} analysis of the data. Therefore, the interaction and interface designs of the application were created incrementally and the design as a whole was added in the Design chapter.\par

The following list outlines the iterations of the prototype including the requirements that were implemented:
\begin{itemize}
	\item Implementation of groups in the population and graph visualizations of certain questions, group\_detail.html and group\_new.html using the residential survey data only (Functional Requirements 1, 2a, 2bi, 2bii, 2biv, 2e)
	\item Addition of survey code translations to the visualizations implemented in the iteration 1.
	\item Implementation of resources\_list and resource\_datafields.html
	\item Implementation of the map in group\_detail.html (Functional Requirements 2iii, 2c)
	\item Implementation of the group\_compare.html (Functional Requirement 2d)
	\item Implementation of clustering and cluster\_detail.html (Functional Requirement 3a)
	\item Implementation of the interface cluster\_stats (Functional Requirement 3b, 3c)
	\item Implementation of the interface cluster\_compare.html (Funcitonal Requirement 3c)
\end{itemize}

The implementation stated with the initial requirements and each succeeding increment involved design, implementation and testing. The functional requirements were relatively done in order it was listed since each requirement usually depended on the previous requirement. The function and interface of each page was done together so that each prototype is functional. The prototype could then be tested and features for the next prototype could be suggested, which follows the spirit of evolutionary prototyping. \par 

Development involved developing sections of the prototype that was independent of the system for some iterations. This was to familiarize myself to the different languages and libraries used in this application, which lead to the creation of an independently functioning section of the prototype. This was then integrated into the system to avoid confusion with other parts of the code. Independently developing it ensured that it was functional before it was integrated. This also reduced the risk of it not working.  An example is the integration of a Google Map and the Google Charts used in group\_detail.html. The clicking of a chart and display on the map was implemented in a separate HTML page and ensured that it worked. This was then integrated into the actual page, group\_detail.html. The user friendliness was also improved through this method, as testing the interfaces in reality makes missing user friendly elements more obvious than could be seen in the design of the interfaces.

\section{Third party libraries}
The third party libraries are listed below:
\begin{itemize}
    \item Django: a Python web framework used to create web applications
    \item Bootstrap: a CSS library used to make webpages more aesthetically pleasing
    \item Pandas (citation): a Python library used for data analysis and data structures
    \item Google Charts: a JavaScript API which creates different types of charts
    \item Google Maps: a JavaScript API which creates a map
    \item Graphos (citation): a Django app which turns data in the form of Python\textquotesingle s list data structure into Google Charts JavaScript code
•	K-modes: a Python library which applies a clustering algorithm most suited for categorical data on a set of data
\end{itemize}
Since it is a web application, a combination of programming languages were used, Python for the back-end with Django, HTML and CSS for the layout aesthetics of the web page and JavaScript to make the pages interactive.


\section{Page implementations}
The subsections below describe how each part of the software was implemented and some screenshots of the actual implementation.

\subsection{Visualization of data into charts}
Pandas Dataframe was used to store the Residential Survey data. The survey data as a CSV was loaded into a Dataframe and the latter was used to query the data around the system. A single instance of the Dataframe was kept to refrain from repetitively opening and reading the same file through the singleton design pattern.\par

Graphos was used to create the Google Charts. The required data of a chart was queried and formatted to Graphos\textsc{\char13} specficiations and a Graphos object was created. Since Django is a web framework, it has a template feature where Python objects, such as a Graphos graph object, could be turned into content for an HTML file.\par

Depending on the type of question, the charts on figure \ref{fig:4charts} was created.

\begin{figure}[h]
\centering
\includegraphics[scale=0.7]{4charts}
\caption{Implementation of the charts}
\label{fig:4charts}
\end{figure}

As a default feature of Google Charts, hovering over a piece of data will display the value of data (see top left of figure \ref{fig:4charts}).

\subsection{Integration of survey code}
Following the design for the preprocessing of data (see figure \ref{fig:figure1}), the English Code translation text file was created. To speed up the creation of this file, the tables were copied into the text file and a small program was created to format the tables, whose elements are spaced by `\textbackslash r', and replaced with a `\textbackslash n' to format the spaces correctly. The rest of the file was typed manually.\par

Readtext.py is the Python module which reads and creates each question into Question objects. Readcsv.py consolidates English and code data into a Python list. This is inputted into a Graphos class which produces the Google Charts JavaScript code. 

\begin{figure}[h]
\centering
\includegraphics[scale=0.5]{figure1.png}
\caption{Implementation of the charts}
\label{fig:figure1}
\end{figure}

Figure \ref{fig:figure1} shows the data flow from raw data to the creation of Graphos charts. This is a simplified view which omits other Django modules.

\subsection{Visualization of wards on a map and ward breakdown chart}

The map was adapted from a tutorial in the Google Maps website which implemented the display of colored areas depending on the value of that area. In this case, the areas were changed to Lambeth\textquotesingle s wards.\par

The interactivity between the Google Map and Google Chart in the group\_detail page was developed as described in Development approach (see section \ref{sec:developmentapproach}). \par

Additional features to the Graphos code was added. A listener to see if the chart was clicked was added to Graphos\textsc{\char13} code. When a chart is clicked, the function which redraws the chart, map and changes the text is invoked. \par

\begin{figure}[h]
\centering
\includegraphics[scale=0.5]{figure2.png}
\caption{Implementation of the charts}
\label{fig:figure1}
\end{figure}

This interactivity required the map\textquotesingle s contents to be changed (i.e. change the colors of the wards depending on the question selected). A JSON containing a list, which is required in the creation of Google Maps, is populated with the selected data from a chart (see figure \ref{fig:figure2}). This JSON is requested by clicking a chart on the right hand side of the page, which invokes the redrawing of the map. The figure below explains the data flow.


\begin{figure}[h]
\centering
\includegraphics[scale=0.3]{figure3.png}
\caption{Screenshot of the group\_detail page}
\label{fig:figure1}
\end{figure}

Hovering over the map\textquotesingle s wards will display a box containing the ward\textquotesingle s name and value on the bottom left hand side of the map (see figure \ref{fig:hover}).

\begin{figure}[h]
\centering
\includegraphics[scale=0.3]{hover}
\caption{Screenshot of hovering over a ward on the Google Map}
\label{fig:hover}
\end{figure}

\subsection{Visualization the Comparison of Clusters}
A group\textquotesingle s data, the whole survey data and clusters\textsc{\char13} data were visualized together for each question in a similar method to the visualization of a single group. To differentiate between each of the clusters\textsc{\char13} data for each question, the cluster id was included into the chart array. This was kept hidden when the array is displayed on the chart, but it is used to know which cluster is clicked when the user wants to display a particular variable on the map.   

\begin{figure}[h]
\centering
\includegraphics[scale=0.3]{comparisonpage}
\caption{Screenshot of the cluster\_compare page}
\label{fig:comparisonpage}
\end{figure}

All charts are stacked bar charts. The click to display on map feature is applicable to these charts as well.

\subsection{Comparison of groups}
On the right-hand side of the page (see figure \ref{fig:groupcompare}), the charts generated from the All group with the same method as in a group\_detail page. However, the left-hand side of the page is a bar chart of the group\textquotesingle s percent difference from the whole survey population. As with a group\textquotesingle s Detail page, clicking a piece of data on a chart will instigate a change in the left-hand side bar chart.\par

The percent difference is calculated using the equation found in the Design chapter (see section \ref{sec:interfacedesign})

\begin{figure}[h]
\centering
\includegraphics[scale=0.3]{groupcompare}
\caption{Screenshot of the group\_compare page}
\label{fig:groupcompare}
\end{figure}

\subsection{Clustering algorithm}

Each group was clustered according to the number of clusters set by the user in the Stats page. The default number of clusters is 1. K-modes clustering (citation) was used since it is more effective than k-means clustering to cluster categorical data, which the survey data is mainly composed of.\par

The results of clustering are stored in the database. The Subcluster Django model created the database table with fields: serial (i.e. serial of row in Survey data), group (i.e. Cluster model) and cluster (i.e. id of cluster). Should the number of clusters be changed by the user, the Subcluster records of the group will be deleted and replaced. If the clusters are queried, then the system will query the Subcluster database table so that the clusters remain the same.

\begin{figure}[h]
\centering
\includegraphics[scale=0.3]{clusterstats}
\caption{Screenshot of the cluster\_stats page}
\label{fig:clustestats}
\end{figure}


\subsection{Resource question page}
Similar to the clusters\_compare page and implemented with a similar Python operation, the survey question page displays more than one group in the same chart. 

\begin{figure}[h]
\centering
\includegraphics[scale=0.3]{surveyquestion}
\caption{Screenshot of the resource\_datafield page}
\label{fig:surveyquestion}
\end{figure}

\section{Testing}
To be continued
\chapter{Professional and Ethical Issues} \label{ch:professionalissues}

The piece of software produced is a combination of my work and the work of Third Parties, which include Open Source libraries such as Django (citation), Graphos (citation), Google Maps (citation), Google Charts (citation), Kmodes (citation), and others (citation). These libraries were utilized to take advantage of existing work however, they were adapted to the purposes of this project. Relevant libraries and sources are cited in both source code and the description in the implementation to give credit to third party content.
The main data used, the 2016 Lambeth Residential Survey, keeps its participants anonymous and according Lambeth has no security issues in terms of publicizing the data. The software places a resident under an anonymous serial, no name of any sort is saved nor a specific address. The software maintains the anonymity of the location of their residence at the Ward level instead of a more specific post code level.\par

The clustering of the survey data used a third-party software. Though sensitive information such as race and sex is included as variables in which to cluster, as far as I know the clustering algorithm used do not discriminate on those variables. \par

During the evaluation stage, a demonstration of the software was conducted to a group of 4 people from the policy department Lambeth Council to gain their feedback. The demonstration and the acceptance of their feedback was done respectfully as they have given honest criticisms of my work. \par

\chapter{Evaluation}

The evaluation of the software involved the comparison between the design to the implementation. Since it is a visualization project, an independent evaluation was taken from a demonstration to some of the potential users from the policy department in Lambeth Council. The method and results of the evaluation is described below.

\section{Demonstration to Lambeth Council}

The demonstration took place in the offices of Lambeth where I presented each page of the web application. The potential of the visualization to aid the formation of strategies were discussed as well as the limitations and improvements of each feature. The requirements were given to them to compare to the actual software as well. The limitations are described below and the solutions to these limitations are in section … of the Conclusion chapter.\par

They have expressed that it has met the objectives overall and that it has the potential given that there would be more features and an option to include data sets to name a few.

The subsections below summarizes the discussions for the visualization pages and the data used.

\subsection{Data used}
The project focuses on a prototype to explore the possibilities of visualizing a population rather than creating a complete tool with usable data. The current survey data could be used as it is to facilitate the initial analysis of the population though the visualization should be taken with caution. \par

They have expressed that the usefulness of the tool relies on the quality and amount of data used. The tool may not be usable at its current state since the survey data is not representative. The current data set used only surveyed 1024 residents of a council of more than 300,000 residents. It is also uses data from only 1 year. The software does not let the user change which annual residential survey to view nor does it utilize past residential surveys.

\subsection{Group\_detail page}
This page gave much information as it not only gave data for each question, but also displaying a piece of data on the map showed where those residents live. This may help in identifying areas which need the most help.\par

In terms of the usefulness of the map, dividing it into wards limits the targeting of a smaller area since wards occupy a large area. Generalizing a ward to behave a certain way may ignore the minorities within the area. Using a smaller geographical division such as the Office of National Statistics\textsc{\char13} (ONS) output area or super output area which cover areas smaller than a ward and is included in the current data can identify more specific areas. \par

Due to the smaller data set, should there only be one resident in a ward with a given survey question, the identification of a more specific area, such as ONS\textsc{\char13} output area, would enable the user to identify if the user. For example, a ward with 1 resident in the disabled group could potentially be a person living in a senior home. This could give insights to people living in senior homes rather than a valid representation of disabled people in that ward.\par

The map\textsc{\char13}s color range was misunderstood as red meant danger and green meant good. The choice of colors could have been constrained to shades of one color.\par

\subsection{Group\_compare page}

This page gave a good indication on which questions are more applicable for which group. For example, the visualization shows more people in long-term illness and acquires child benefits are more likely to say they are not paid the London Living Wage. This could potentially help in identifying which resources should be given to which groups of people.\par

The information is limited to the percentage of a piece of data in relation to the data of a question. The comparison between groups and clusters are also in the form of percentages. Other statistical data such as mean, mode, median and standard deviation could be integrated which results in deeper analysis.

\subsection{Clusters pages}

Clustering in a statistical sense, needs to be explained to the users. Upon demonstrating the tool to the council workers, the concept of clustering needed to be explained. The visualization of the clusters as seen in cluster\_detail or cluster\_compare could only be understood given that clustering.
The cluster\_stats page was confusing as the table of means were still in the survey code. There needed to be English translations of the code supplemented. This was fixed after the demonstration where hovering over the question will show the English translations.

\subsection{Resource\_datafield page}
They were interested in this page as it has the potential to confirm or debunk some of their expectations of some groups. For example it confirmed that the ethnicities of people living in residential homes were from English, Caribbean or African descent. Another application could be how they would approach some of the groups using the questions regarding the medium of contact with the council or how they access the internet.

\section{Project evaluation}

Though the tool was inspired by the problems faced by a council based on the interview with a Lambeth council worker, the requirements and design of the tool was up to me through research into other visualization applications. Therefore, the end product may not be suitable for them. However, they did express that it has met the requirements I have set for the software. Regular contact during the whole process in terms of feedback of the design and usability, and more specific data sets would have produced a more usable product for their needs.

\chapter{Conclusion and Future Work}

This aim of this project was to create a tool that will aid the formation of strategies to help segments of a population by a local council. The end product was a Django web application which gives different perspectives through different visualizations of a single data set. The tool\textsc{\char13}s usefulness depends on the quality of the data and whether the data represents the population well. According to some members of the policy department in Lambeth Council, it has potential to be used should there be more thought put in what data sets are to be used.\par

The tool may have applications for other local governments dealing who would like to get to know a population more and explore what sorts issues are present within segments of the population. Outside of local government, this could be used for anyone looking to do customer segmentation such as businesses who want to tailor their services to the surrounding residents.

\section{Future work}
\begin{itemize}
	\item The tool is implemented as a Django web application and should there be improvements, the tool could be packaged as such. This could enable the developers in local councils or potential users of the tool to implement it themselves using their own data sets and explore their population accordingly. Some pages could be turned into reusable Django Views class that can input a data set and output a page with the visualizations which could be used in other Django applications.
	\item The user could benefit with more flexibility in terms of which questions are to be included as the only certain questions are shown and which data set is to be used. Including other years\textsc{\char13} data sets could show whether a segment is improving or not.
	\item Utilizing different sizes of geographical divisions such as ONS\textsc{\char13} output areas or super output areas instead of wards could be beneficial in targeting more specific areas.
	\item In other applications of the tool, the data may be more sensitive and require security features.
	\item The comparison of the results on two different questions as graphs or maps could be added as another page. This would enable the user to compare questions rather than groups. It could highlight correlations between clusters as well. For example, a map showing the residents who claim child benefits and another map showing the residents who live in a council flat might show the same trends and could be interpreted as a correlation between the two.
	\item The use of a more performance database such as MongoDB to store the data set could be implemented in future iterations of the program to increase scalability for data sets larger than one those rows.
More statistical information such as mean, median, mode and standard deviation which standard knowledge in local government will provide more insight to the distribution of the data.
\end{itemize}

%%%%%%%%%%%%%%%%%%%%%%%%%%%%%%%%%
% References
%%%%%%%%%%%%%%%%%%%%%%%%%%%%%%%%%
\bibliographystyle{vancouver}
\bibliography{references}
\addcontentsline{toc}{section}{Bibliography}

%%%%%%%%%%%%%%%%%%%%%%%%%%%%%%%%%
% Appendices
%%%%%%%%%%%%%%%%%%%%%%%%%%%%%%%%%
\appendix
%\include{Appendices/appendix}
\chapter{User Guide}

This appendix provides a walk-through of how to run and use the web application.

\section{Running the program}

The requirements before running the program are:

\begin{itemize}
	\item Python 2.7
	\item Django 1.10.5
	\item pandas 0.19.2
	\item django-graphos 0.3.23
	\item sklearn 0.0
	\item numpy 1.12.0
	\item scipy 0.18.1
	\item kmodes 0.7
\end{itemize}

Open terminal, go to the appropriate directory and enter the following:

\begin{lstlisting}
python manage.py runserver
\end{lstlisting}

Then open a browser and enter the following url:\par

\begin{lstlisting}
http://localhost:8000
\end{lstlisting}

The Home page should appear and you are ready to use the application.

\section{Home page}

Figure \ref{fig:homepage} shows an example of a home page. The navigation bar provides the main functions of the software. The page lists the groups which you have created, if you have not created a group yet it will be blank.

\begin{figure}[h]
\centering
\includegraphics[scale=0.3]{homepage}
\caption{Screenshot of the homepage}
\label{fig:homepage}
\end{figure}

\section{Creating a group}

Begin by clicking on the ``Create group'' button on the navigation bar. The page in figure \ref{fig:creategrouppage} will appear. Enter the details for the group and then click ``Submit''. Ensure that the name of the group is not the same as other group names and each factor is filled out or else an error will appear.\par 

Select ``Any of the above'' if you do not wish to put anything for a factor.\par

\begin{figure}[h]
\centering
\includegraphics[scale=0.3]{creategrouppage}
\caption{Screenshot of the Create group page}
\label{fig:creategrouppage}
\end{figure}

View the group you have created by going to the home page or selecting it under ``View group'' in the navigation bar. The group detail page will appear (see section \ref{sec:groupdetailpage})

\section{Group pages} 

\subsection{Group detail page}
\label{sec:groupdetailpage}

The visualizations for the group could on this page. On the left side of the page shows the map divided into wards and Ward breakdown chart. The initial state of the map shows the population in each ward of the selected group. 

Hover over the map\textsc{\char13}s wards to view the name and value of the ward. The range above the map shows the meaning of the colors of the wards.

On the right side of the page are the charts of selected questions from Lambeth\textsc{\char13}s 2016 Residential Survey and the following buttons:

\begin{itemize}
	\item ``Refresh map'': Click to change the map to its default state i.e. showing the population of the current group.
	\item ``Compare clusters'': Click to compare data for each of the questions shown between data on the group\textsc{\char13}s clusters, the group and whole population.
	\item ``Stats'': Click to change the number of the group\textsc{\char13}s clusters and view the table of means for each question and cluster.
	\item ``Cluster n'': Click to view the visualization for this particular cluster number n.
	\item ``About group'': Click to view information about the factor setting, total population of the group, the source of the data and the number of clusters. There is also a delete button should you wish to delete this group.
	\item ``Edit group'': Click to edit the group\textsc{\char13}s name and factor settings.
	\item ``i'': Click to see instructions for this page.
\end{itemize}

\begin{figure}[h]
\centering
\includegraphics[scale=0.3]{figure3.png}
\caption{Screenshot of the Group detail page}
\label{fig:groupdetailpage}
\end{figure}

Some percentages are shown on the charts. Hover over the charts to view a tool tip of the percentage and original value of a variable. If in the legend, the whole text is not shown, hover over it and a tool tip of the whole text will appear.

\subsection{Compare clusters page}

This page seen in figure \ref{fig:compareclusterspage} allows the user to compare the results of each question between the whole population, the group and all of the group\textsc{\char13}s clusters. \par

The functionality of this page is similar to the Group detail page (see see \ref{sec:groupdetailpage})

\begin{figure}[h]
\centering
\includegraphics[scale=0.3]{comparisonpage}
\caption{Screenshot of the Compare clusters page}
\label{fig:compareclusterspage}
\end{figure}

\subsection{Stats page}

This page seen in figure \ref{fig:statspage} allows the user to change the number of clusters for this group. Clicking the up or down arrow will increase or decrease, respectively, the number of clusters. The system will not allow a number less than 1.\par

The Elbow chart aids the user in finding the optimal number of clusters using the elbow method. The biggest bend in the graph gives the optimal number of clusters. In figure\ref{fig:statspage} the number of clusters is 3. Choosing the bend tends to be subjective and may not always be clear which bend to choose. \par

\begin{figure}[h]
\centering
\includegraphics[scale=0.3]{statspage}
\caption{Screenshot of the Stats page}
\label{fig:statspage}
\end{figure}

Below the chart is the Table of mean values. The columns are the questions and the rows are the clusters. Therefore, values indicate the mean value for a question and cluster. This enables the user to identify differences between each cluster. For example see figure \ref{fig:tableofmeans}. Cluster 2 shows there may be a relation between Asian ethnicity (QETH), since the mean is 11 meaning Indian, and lower rating for belonging to the neighborhood (Q13\_R1), since the mean is around 2, compared to the other clusters, since their mean is around 1. 1 means they strongly agree that they belong to the neighborhood. 2 means they just agree. To see the distribution in more detail go to the Cluster detail page.

\begin{figure}[h]
\centering
\includegraphics[scale=0.3]{tableofmeans}
\caption{Screenshot of the Table of means in the Stats page}
\label{fig:tableofmeans}
\end{figure}

Hovering over the question codes (e.g. Q11) will display the code translations above the table (see figure \ref{fig:tableofmeans}).

\subsection{Cluster detail page}

Clicking the ``Cluster n'' button, n is a number 	greater than 0, will show the visualizations for that cluster. The same functionalities described in section \ref{sec:groupdetailpage} are available on this page.

\section{Compare groups page}

Clicking the ``Compare groups'' button on the navigation bar will show the page shown in figure \ref{fig:groupcompare}.\par

\begin{figure}[h]
\centering
\includegraphics[scale=0.3]{groupcompare}
\caption{Screenshot of the Compare groups page}
\label{fig:groupcompare}
\end{figure}

On the left side of the page, the comparison of the groups could be seen as a chart. Groups on the right side of the line mean that they have a bigger proportion than the whole population which have the selected data variable.\par

On the right side of the page, the visualizations for the All group is displayed. Clicking the charts will display the selected data variable on the comparison chart on the left side of the page.\par

\section{Survey pages}

\subsection{Survey page}

Click on the ``Survey'' button on the navigation bar to go to this page (see figure \ref{fig:surveypage}). This page shows the links to the questions used in the system. Click a question and the Survey question page will appear (see section \ref{sec:questionpage}).

\begin{figure}[h]
\centering
\includegraphics[scale=0.3]{surveypage}
\caption{Screenshot of the Survey page}
\label{fig:surveypage}
\end{figure}

\subsection{Survey question page} \label{sec:questionpage}

This page (see figure \ref{fig:surveyquestion}) visualizes the data of all groups on a single question. If the question is a multiple answer question, each subquestion is presented as a separate graph. This visualization is useful to see the differences in the proportions of each variable in each group.

\begin{figure}[h]
\centering
\includegraphics[scale=0.3]{surveyquestion}
\caption{Screenshot of the Survey question page}
\label{fig:surveyquestion}
\end{figure}
\chapter{Source Code}

The following pages are the Table of Contents for the source code listings and the source code themselves.
\end{document}
